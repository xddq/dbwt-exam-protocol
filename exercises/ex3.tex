\subsection{Aufgabe 3: Normalisierung}
\label{sec:ex3}
\begin{enumerate}[label=\alph*)]
        % give item a label to point to it in the caption
    \item  Überführen Sie die folgende Tabelle in die 1NF.\label{itm:a3-first}\\
        %\\[2ex]  skips two lines
        \begin{table}[h!]
            \centering
            \begin{adjustbox}{max width=\textwidth}
                \begin{tabular}{*{4}{|c}|} % creates table with 4 columns
                    \hline % draws a horizontal line
                    \textbf{LieferID} & \textbf{Anschrift} & 
                    \textbf{Organisationsname} & \textbf{Lieferzeitpunkt}\\
                    \hline
                    1 & 52070; Eupener Straße; 70 & Fh-Aachen & Vormittags; 
                    Nachmittags\\
                    \hline
                \end{tabular}
            \end{adjustbox}
            \caption{Ausgangstabelle für \ref{itm:a3-first}}
            \label{tab:transfrom_to_1nf}
        \end{table}
    \item Überführen Sie folgende Tabelle direkt in die 3NF. Unterstreichen Sie
        jeweils den \underline{Primärschlüssel} und den \dashuline{Fremdschlüssel}
        .\label{itm:a3-second}\\
        \begin{table}[h!]
            \centering
            \begin{adjustbox}{max width=\textwidth}
                \begin{tabular}{*{8}{|c}|} % draws seven columns
                    \hline
                    \textbf{Hersteller} & \textbf{Farbe} & \textbf{Knz} & \textbf{Farbcode}
                    & \textbf{Herstellersitz} & \textbf{Fahr\_Nr} & \textbf{Fahr\_Vorname}
                    & \textbf{Fahr\_Nachname}\\
                    \hline
                    Opel & Silber & DB-WT-10 & 135 & Rüsselsheim & 13337 & Ronnie & Nator\\
                    \hline
                    Opel & Blau & DB-WT-11 & 274 & Rüsselsheim & 13338 & Mike & Mann\\
                    \hline
                    VW & Beige & DB-WT-12 & 271 & Wolfsburg & 13339 & Frauke & Frau\\
                    \hline
                \end{tabular}
            \end{adjustbox}
            \caption{Ausgangstabelle für \ref{itm:a3-second}}
            \label{tab:transform_to_3nf}
        \end{table}
    \item Welche Bedingungen müssen gelten, damit eine Tabelle in der 2NF vorliegt?
    \item Erklären Sie den Begriff der funktionellen Abhängigkeit und erläutern
        Sie diesen an einem Beispiel.
\end{enumerate}
