\newpage
\subsection{Aufgabe 4: SQL}
\label{sec:Aufgabe4}
Gegeben sind folgende Tabellen: (Diese entsprechen nicht exakt der Vorgabe aus der Klausur! Insbesondere die Attributwerte sind nachträglich hinzugefügt.)
\\[0.2cm] %sets spacing between these two lines to 0.2cm

\begin{tabular}{c|c}
     \underline{InterpretID} & Interpret \\
     \hline
     1 & Anastacia \\
     2 & Pink Floyd \\
     3 & Beatles
\end{tabular}

\begin{tabular}{c|c|c|c}
     \underline{AlbumID} & Name & \udensdash{InterpretID} & Erscheinungsdatum\\
     \hline
     1 & Not That Kind  & 1 & 2000 \\
     2 & Freak of Nature & 1 & 2001 \\
     3 & Wish You Were Here & 2 & 1975
\end{tabular}

\begin{tabular}{c|c|c|c}
     \underline{TrackID} & Trackname & \udensdash{AlbumID} & Duration\\
     \hline
     1 & Not That Kind  & 1 & 200\\
     2 & I’m Outta Love  & 1 & 350\\
     3 & Cowboys \& Kisses & 1 & 180 \\
     4 & Shine On You Crazy Diamond & 2 & 200 \\
     5 & Paid my Dues & 2 & 190
\end{tabular}
\begin{enumerate}[label=\alph*)]
    \item Geben Sie alle Tracks mit Duration > 200 aus.
    \item Geben Sie die Länge jedes Albums (Summe über alle Titel des Albums) aus.
    \item Geben Sie alle Interpreten und - sofern vorhanden - auch die zugehörigen Alben aus.
    \item Geben Sie alle Interpreten aus, zu denen es kein Album gibt.
    \item Geben Sie alle Tracks aus, deren Länge größer als der Durchschnitt ist.
    \item Erzeugen Sie eine View '5laengstetracks', welche die 5 längsten Tracks des Albums mit der ID 1 ausgibt.
    \item Löschen Sie die erzeuge View.
\end{enumerate}
