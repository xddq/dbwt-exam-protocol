\newpage
\subsection{Aufgabe 5: ER-Diagramm}
\label{sec:Aufgabe5}
\begin{enumerate}[label=\alph*)]
    \item Erstellen sie ein ER-Diagramm nach Chen, \textbf{ohne} Beziehungen, welches die
        folgende Beschreibung zusammenfasst.\\[2ex] In einem Autohaus werden Autos
        verkauft. Die Autos besitzen Motoren unterschiedlicher Klassen. Es gibt
        kleine, mittlere und große Motoren. Zu jedem Auto wird eine Fahrgestellnummer
        gespeichert, die das Auto eindeutig identifziert. Ein genaues Gewicht wird
        ebenfalls gesichert. Zu jedem Auto wird der passende Verbrauch berechnet,
        welcher sich aus dem Gewicht und der größe des Motors zusammensetzt. Es
        gibt jedes Auto mit unterschiedlichen Ausstattungen (Klimaanlage, Tempomat,
        Sitzheizung etc..). Bei jedem Hersteller ist die Adresse und der damit
        verbundene Ort, PLZ, die Straße und die Hausnummer hinterlegt.
    \item Erstellen Sie zu folgendem Sachverhalt ein ER-Diagramm nach Chen
        \textbf{mit} Verwendung von Beziehungen.\\[2ex]
        Im folgenden Text wird das Geschäftsmodell einer Carsharing Firma erläutert.
        Für diese sollen Sie ein ER-Modell nach Chen mit Beziehungen anfertigen!
        In der Firma soll es möglich sein, dass jeder Mieter auch mehrere Autos
        mietet. Beim Mieten wird der Start- und Endzeitpunkt festgelegt. Die Autos
        werden in regelmäßgen Abständen gereinigt werden müssen. Dazu kommen Sie
        in die Reinigung. Die Reinigung ist groß und in viele Hallenbereiche
        unterteilt. Für jede Reinigung wird das Datum gespeichert. Es passen
        mehrere Autos in einen Hallenbereich und täglich werden mehrere Autos
        gereinigt. Mieter müssen sich mit ihrer Mail Adresse sowie dem Namen
        registrieren. Durch diese Parameter sind sie eindeutig identifizierbar.
    \item
        Erstellen Sie zu folgendem Sachverhalt ein ER-Diagramm mit einer Min-Max
        Kardinalität\\[0.5cm]
        Ein Nutzer muss entweder Anbieter oder Mieter sein. Er kann allerdings auch
        beide Rollen einnehmen. Nutzer besitzen eine eindeutige KundenNr. Sie
        müssen einen Username und ihren vollen Namen angeben. Anbieter müssen
        ihre Adresse und ihre Bankinformation hinterlegen. Optional kann eine
        Bewertung für Mieter gespeichert werden.

    \item Notieren Sie die Lösung aus Aufgabe (c) in der Relationsschreibweise \textit{attribut(\underline{wert1},...))}.
\end{enumerate}
