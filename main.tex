\documentclass[DIV=16,parskip]{scrartcl} % siehe <http://www.komascript.de>
\usepackage{selinput} % Eingabecodierung automatisch ermitteln
\SelectInputMappings{ % siehe <http://ctan.org/pkg/selinput>
  adieresis={ä},
  germandbls={ß}
}
\setcounter{secnumdepth}{0}
\usepackage[ngerman]{babel}% Das Beispieldokument ist in Deutsch,
                % daher wird mit Hilfe des babel-Pakets
                % über Option ngerman auf deutsche Begriffe
                % und gleichzeitig Trennmuster nach den
                % aktuellen Rechtschreiberegeln umgeschaltet.
                % Alternativen und weitere Sprachen sind
                % verfügbar (siehe <http://ctan.org/pkg/babel>).
								
% Zeichensatz Latin Modern
\usepackage{lmodern}
\renewcommand*\familydefault{\sfdefault} %% Only if the base font of the document is to be sans serif
\usepackage[T1]{fontenc}

% Für Grafiken
\usepackage{graphicx}
\usepackage[absolute]{textpos}

% Für Formatierung von Tabellen
\usepackage{tabu}

% Für Mathematik
\usepackage{amsmath}

% Für mehr Flexibilität bei Aufzählungen
\usepackage{enumerate}
\usepackage{enumitem}

\usepackage{minted}

\usepackage{tikz}
\usetikzlibrary{arrows,automata}

\begin{document}

% FH Aachen Logo nach rechts oben in die Ecke der ersten Seite
%\begin{textblock*}{15mm}[1,0](210mm,10mm)
%	\includegraphics[width=15mm]{img/logo.png}
%\end{textblock*}

\begin{flushleft}
{\footnotesize{Prof.\,Dr.\,A.\,Hannig \textbar{ }FH Aachen}} \\

\vspace{24pt}
\parindent=-0.5pt
\begin{tabu} {@{}X[l]X[c]X[r]@{}}
{\LARGE{\textbf{Datenbanken und Webtechnologien}}} & {\Large{WS1920}} & {\textnormal{Klausur 09.07.2020}} \\
\end{tabu}
\end{flushleft}

\vspace{24pt}

\newcommand{\udensdash}[1]{%
    \tikz[baseline=(todotted.base)]{
        \node[inner sep=1pt,outer sep=0pt] (todotted) {#1};
        \draw[densely dashed] (todotted.south west) -- (todotted.south east);
    }%
}%

\subsection{Aufgabe 1: HTML}
\label{sec:Aufgabe1}
\begin{enumerate}[label=\alph*)]
    \item Was ist ein Pseudoelement? (\textbf{multiple chioce})
    \item Wie sehen Kommentare in HTML? (\textbf{multiple chioce})
    \item Welche Sonderzeichen müssen in der URL ersetzt werden? (\textbf{multiple chioce})
        \begin{enumerate}[label=\arabic*.]
        \item =
        \item \#
        \item ?
        \item \%
        \end{enumerate}
    \item Wie sieht der zugehörige HTML-Code aus? \\
        \includegraphics[width=15cm]{img/klausur.JPG}
        \begin{enumerate}[label=\arabic*.]
                % use '' instead of " because "o -> ö etc. in babel package.
            \item Die Parameter sollen als ''benutzername'', ''passwort'' und
                ''sprache'' beim Server abfragbar sein.
            \item Benutzername ist max. 80 Zeichen lang und muss immer angegeben werden.
            \item Passwort soll nicht lesbar sein.
            \item Passwort vergessen leitet nach ''/passwortzuruecksetzen'' weiter.
            \item Sprache Deutsch soll standardmäßig aktiv sein. Die andere Option
                ist Englisch.
            \item Das Formular soll an ''/auswertung.php'' geschickt werden und
                POST ausgelesen werden können.
         \end{enumerate}
    \item Nennen Sie 4 Statuscodes und erläutern Sie kurz ihre Bedeutung.
    \item Nennen Sie 4 HTTP Methoden und erläutern Sie kurz ihre Bedeutung.
\end{enumerate}

\subsection{Aufgabe 2: PHP}
\label{sec:Aufgabe2}
\begin{enumerate}[label=\alph*)]
    \item Gegeben ist folgendes Array in PHP. Schreiben Sie eine Funktion, so dass der angegebene HTML-Code dabei heraus kommt.
        \begin{minted}{php}
            array = [
                1 => 'a',
                2 => ['b', 'c'],
                3 => [4, 5, 6]
            ];
        \end{minted}
    
        \begin{minted}{html}
            <ol>
                <li> 1: a </li>
                <li> 2:
                    <ul>
                        <li> b </li>
                        <li> c </li>
                    </ul>
                </li>
                <li> 3:
                    <ul>
                        <li> 4 </li>
                        <li> 5 </li>
                        <li> 6 </li>
                    </ul>
                </li>
            </ol>
        \end{minted}
    \item Mit welchen Tag beginnt ein PHP-Script? (\textbf{multiple chioce})
    \item Wie inkludiere ich in PHP ein File? (\textbf{multiple chioce})
    \item Wie erhalte ich den HTTP-Header??? (\textbf{multiple chioce})
        \begin{enumerate}[label=\arabic*.]
            \item GET
            \item REQUEST
            \item POST
        \end{enumerate}
\end{enumerate}

\subsection{Aufgabe 3: Normalisierung}
\label{sec:Aufgabe3}
\begin{enumerate}[label=\alph*)]
    \item in 1NF (Lieferzeitpunk Anschrift(Ort, PLZ) NameORG ID )
    \item Hersteller - Farbe - Kennzeichen - Farbcode - Herstellersitz - Fahr\_Nr - Fahr\_Vorname - Fahr\_Nachname in 3NF 
    \item Wann einer Tabelle in 2NF
    \item Was ist funtionale Abhängigkeit
\end{enumerate}

\newpage
\subsection{Aufgabe 4: SQL}
\label{sec:Aufgabe4}
Gegeben sind folgende Tabellen: (Diese entsprechen nicht exakt der Vorgabe aus der Klausur! Insbesondere die Attributwerte sind nachträglich hinzugefügt.)
\\[0.2cm] %sets spacing between these two lines to 0.2cm

\begin{tabular}{c|c}
     \underline{InterpretID} & Interpret \\
     \hline
     1 & Anastacia \\
     2 & Pink Floyd \\
     3 & Beatles
\end{tabular}

\begin{tabular}{c|c|c|c}
     \underline{AlbumID} & Name & \udensdash{InterpretID} & Erscheinungsdatum\\
     \hline
     1 & Not That Kind  & 1 & 2000 \\
     2 & Freak of Nature & 1 & 2001 \\
     3 & Wish You Were Here & 2 & 1975
\end{tabular}

\begin{tabular}{c|c|c|c}
     \underline{TrackID} & Trackname & \udensdash{AlbumID} & Duration\\
     \hline
     1 & Not That Kind  & 1 & 200\\
     2 & I’m Outta Love  & 1 & 350\\
     3 & Cowboys \& Kisses & 1 & 180 \\
     4 & Shine On You Crazy Diamond & 2 & 200 \\
     5 & Paid my Dues & 2 & 190
\end{tabular}
\begin{enumerate}[label=\alph*)]
    \item Geben Sie alle Tracks mit Duration > 200 aus.
    \item Geben Sie die Länge jedes Albums (Summe über alle Titel des Albums) aus.
    \item Geben Sie alle Interpreten und - sofern vorhanden - auch die zugehörigen Alben aus.
    \item Geben Sie alle Interpreten aus, zu denen es kein Album gibt.
    \item Geben Sie alle Tracks aus, deren Länge größer als der Durchschnitt ist.
    \item Erzeugen Sie eine View '5laengstetracks', welche die 5 längsten Tracks des Albums mit der ID 1 ausgibt.
    \item Löschen Sie die erzeuge View.
\end{enumerate}

\newpage
\subsection{Aufgabe 5: ER-Diagramm}
\label{sec:Aufgabe5}
\begin{enumerate}[label=\alph*)]
    \item Erstellen sie ein ER-Diagramm nach Chen, \textbf{ohne} Beziehungen, welches die
        folgende Beschreibung zusammenfasst.\\[2ex] In einem Autohaus werden Autos
        verkauft. Die Autos besitzen Motoren unterschiedlicher Klassen. Es gibt
        kleine, mittlere und große Motoren. Zu jedem Auto wird eine Fahrgestellnummer
        gespeichert, die das Auto eindeutig identifziert. Ein genaues Gewicht wird
        ebenfalls gesichert. Zu jedem Auto wird der passende Verbrauch berechnet,
        welcher sich aus dem Gewicht und der größe des Motors zusammensetzt. Es
        gibt jedes Auto mit unterschiedlichen Ausstattungen (Klimaanlage, Tempomat,
        Sitzheizung etc..). Bei jedem Hersteller ist die Adresse und der damit
        verbundene Ort, PLZ, die Straße und die Hausnummer hinterlegt.
    \item Erstellen Sie zu folgendem Sachverhalt ein ER-Diagramm nach Chen
        \textbf{mit} Verwendung von Beziehungen.\\[2ex]
        Im folgenden Text wird das Geschäftsmodell einer Carsharing Firma erläutert.
        Für diese sollen Sie ein ER-Modell nach Chen mit Beziehungen anfertigen!
        In der Firma soll es möglich sein, dass jeder Mieter auch mehrere Autos
        mietet. Beim Mieten wird der Start- und Endzeitpunkt festgelegt. Die Autos
        werden in regelmäßgen Abständen gereinigt werden müssen. Dazu kommen Sie
        in die Reinigung. Die Reinigung ist groß und in viele Hallenbereiche
        unterteilt. Für jede Reinigung wird das Datum gespeichert. Es passen
        mehrere Autos in einen Hallenbereich und täglich werden mehrere Autos
        gereinigt. Mieter müssen sich mit ihrer Mail Adresse sowie dem Namen
        registrieren. Durch diese Parameter sind sie eindeutig identifizierbar.
    \item
        Erstellen Sie zu folgendem Sachverhalt ein ER-Diagramm mit einer Min-Max
        Kardinalität\\[0.5cm]
        Ein Nutzer muss entweder Anbieter oder Mieter sein. Er kann allerdings auch
        beide Rollen einnehmen. Nutzer besitzen eine eindeutige KundenNr. Sie
        müssen einen Username und ihren vollen Namen angeben. Anbieter müssen
        ihre Adresse und ihre Bankinformation hinterlegen. Optional kann eine
        Bewertung für Mieter gespeichert werden.

    \item Notieren Sie die Lösung aus Aufgabe (c) in der Relationsschreibweise \textit{attribut(\underline{wert1},...))}.
\end{enumerate}

\subsection{Aufgabe 6: (XML)}
\label{sec:Aufgabe6}
\begin{enumerate}[label=\alph*)]
    \item \textit{Hier kommt die Aufgabe}
\end{enumerate}
\subsection{Aufgabe 7: (Serialisierbarkeit)}
\label{sec:Aufgabe7}
\begin{enumerate}[label=\alph*)]
    \item \textit{Hier kommt die Aufgabe}
    Deadlocks:
Was sind ISO-Level
Wie wird das in SQL aktiviert
SET TRANSACTION ISOLATION LEVEL = curly
REPEATABLE READ | READ COMMITTED |
READ UNCOMMITTED | SERIALIZABLE
curly
Was ist Deadlock
Situation in Deadlock
Dirty Read darstellen
Serialisierbarkeit Konfliktgraph und -Menge
\end{enumerate}

\end{document}
